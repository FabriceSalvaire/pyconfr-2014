\documentclass[12pt]{article}

%** Package *********************************************************************

%**** Page settings *********************

\usepackage[%
paper=a4paper,%
includeheadfoot,%
margin=2cm,%
headsep=0cm, footskip=0cm,%
]{geometry}

\usepackage[T1]{fontenc}
\usepackage{lmodern} % Postscript layout of CM with T1 coding

\begin{document}

\section{Slide 1}

Bonjour, 

Je vais vous présenter une interface expérimentale pour lier OpenGL à Python, \\
que j'ai réaliser au cours de cette année.

\section{Slide 2}

Ma présentation va s'articuler en trois temps~: \\
  dans un premier temps je vais introduire l'API OpenGL et les Gpus, \\
  ensuite je vais vous parler de l'interface \\
  et finir par une démonstration.

\section{Slide 3}

Pour commencer nous allons regarder l'évolution des processeurs graphiques au cours du temps,

en 1970 les pionniers construisaient de grosses machines, \\
ensuite les stations de travail sont apparues, \\
puis les accélérateurs graphiques se sont démocratisés, \\
en 2000 Nvidia a commercialisé le premier GPU, \\
puis en 2006 sont apparus les processeurs génériques.

On observe une convergence, il n'y a plus de différence entre le monde professionnel et grand-publique, \\
la même technologie est utilisé de l'embarqué aux super-calculateurs.

\section{Slide 4}

Qu'est ce qui différencie un GPU d'un CPU ?

Ce n'est pas la même manière d'exploiter la surface de silicium comme vous le voir sur ces photos de
dies.

Il s'agit ...

\section{Slide 5}

À quoi servent les GPUs ?

On peut les utiliser pour la 3D, dans les moteurs de jeu, la CAO, la visualisation scientifique, et
plus récemment pour les interfaces graphiques et même la 2D.

\section{Slide 6}

L'API OpenGL est un standard ouvert gouverné par le groupe khronos qui permet de -piloter- un GPU. \\
et la seule API multi-platforme. \\
C'est l'API pour Linux et Android. \\
Il y a plusieurs API, orientée desktop, OpenGL ES pour l'embarqué et WebGL pour le web.

OpenGL n'est pas un monde idéal, mais il y a des initiatives pour une nouvelle génération.

\section{Slide 7}

OpenGL a été créé par Silicon Graphics il y a plus de 20 ans déjà. En 2008 son développement s'est
accéléré avec l'avènement du pipeline programmable livré avec la V3. Mais il y a des concurrents.

Comme souvent on observe un déphasage due à l'inertie au changement. Il y a un déphasage entre
OpenGL, le matériel, le parc machine, et le programmeur.

Lorsqu'on développe une application avec OpenGL, une question fondamental est de savoir qu'elle API choisir.

\section{Slide 8}

L'API OpenGL est découpé en extensions où chacune regroupe un ensemble de nouvelle fonctions avant
leurs intégrations dans l'API core.

Mesa est à l'origine une implémentation logicielle d'OpenGL, aujourd'hui c'est un composant
essentiel pour OpenGL sous Linux pour les drivers open source.

Le graphique représente le statu(s) actuel de Mesa pour le GPU d'Intel. \\
On voit que l'on est à deux doigts d'OpenGL V4.

Aujourd'hui seul le driver propriétaire de Nvidia implémente la quasi totalité de l'API OpenGL.

\section{Slide 9}

Aujourd'hui OpenGL se résume à un langage de calcul générique GLSL qui est mis en oeuvre dans des programmes
appelés shader et des fonctions spécifiques au graphisme.

Les paramètres d'un shader sont des constantes appelés uniform qui peuvent être des flottants, des
matrices, ou des indexes représentant une texture. Les données variables d'un shader sont des
vecteurs de une à quatre dimensions.

Ces vecteurs sont issues d'un flux de donnés qui représente des vertex de primitive graphiques tel
que des lignes et des triangles qui sont interpolés en pixels.  

\section{Slide 10}

Il y a plusieurs solution pour interfacer du code C à Python, \\
chacune à ses avantages et ses inconvénients.

Il y a deux grandes familles, celles qui sont basé sur l'API C Python et celles qui sont basé sur la
bibliothèque FFI.

Ensuite il y a des solutions qui sont directement capable de lire les fichiers d'en-tête C, \\
et des solutions qui ne requiert pas de compilateur C.

Au niveau de l'interpréteur Pypy (JIT), seul Cffi est supporté, elle est d'ailleurs développée par
les mêmes personnes.

En résumé, ...

\section{Slide 11}

PyOpenGL est la référence actuelle pour utiliser OpenGL en Python.

Ce package en est à sa troisième génération, \\
au début il s'agissait d'une extension basé sur l'API C, \\
puis cela a été automatisé grâce à SWIG, \\
et par la suite réimplémenté en utilisant ctypes et parser de fichiers d'en-tête.

La V3 a conservé une compatibilité ascendante avec la V2

En outre il existe des alternatives plus spécifique, \\
Pyglet pour les jeux, et plus récemment vispy pour la visualisation scientifique.

(monté parser, supprimer Pyopengl ...)

\section{Slide 12}

Au cours de ces dernières années, j'ai utilisé PyOpenGL pour développer une application
industrielle.

J'ai rencontré au fil du temps plusieurs difficultés, essentiellement due à un déphasage avec OpenGL. \\
Des constantes étaient absentes, \\
Certaines fonctions n'ont pas de Pythonic Wrapper et requiert du ``glue code'' pas très sympathique. \\
Des fonctions étaient perdues dans la hiérarchie de module, et retrouvé à coups de grep dans le code source, \\
qui est composé d'au moins 2000 fichiers, ce qui n'aide pas pour comprendre comment fonctionne PyOpenGL.

Tout cela peut se -comprendre-, car PyOpenGL est un projet ancien \\
alors qu'OpenGL et Python ont beaucoup évolué depuis. \\
Une autre difficulté est qu'il n'y a pas d'API de haut niveau pour l'OpenGL moderne. \\
(expliqué en quoi c'est gênant)

(en résumé)

\section{Slide 13}

À quoi ressemble l'API OpenGL?

La dernière version d'OpenGL comporte pas moins de 1328 constantes et 653 fonctions.

% /fondamentales
Les paramètres des fonctions sont des types de bases définies via des typedefs et des
pointeurs. Les valeurs retournées sont en général simple à part quelques cas.

Du point de vue de Python, les paramètres sont donc des entiers, des flottants, des chaînes de
caractères et des tableaux Numpy qui sont en réalité des tableaux C. % mieux ?

% la manière efficace de stocker les données.

\section{Slide 14}

En Juin 2013, le groupe Khronos a mis à disposition du publique, un fichier XML qui décrit
l'ensemble de l'API OpenGL, les constantes, les fonctions et leur prototypes, les extensions, ainsi
que la composition de chaque version et profile. % définir profile

Ce fichier apporte des informations essentielles par rapport au seul fichier d'en-tête, comme le
paramètre correspondant à la taille d'un tableau.

\section{Slide 15}

Nous allons à présent regarder de plus près les prototypes des fonctions.

Il y a cinq type de paramètres.

Les paramètres simples qui sont passé par copie.

Les paramètres qui sont en faîte des valeurs retourné, qui doivent être passé en C par référence
lorsqu'il y a plus d'une valeur retournée. On les reconnaît à leur taille unitaire.

Les tableaux en écriture ou lecture passés sous forme de pointeur, on en principe en C un paramètre
indiquant la taille.

Les tableaux en lecture seule sont reconnaissable par l'usage du qualificatif ``const''.

Enfin il y a les pointeurs dit complexe pour lequel la taille est calculé à partir d'un ou plusieurs paramètres.

\section{Slide 16}

L'idée de départ de cette interface est d'utiliser XML registry pour générer automatiquement une
interface pour OpenGL.

L'architecture repose sur une pile de trois composants: \\
  une interface orientée objet de l'XML Registry dit ``GL API'', \\
  un ``wrapper'' basé sur ctypes et cffi qui repose sur la ``GL API'', \\
  et une API de haut niveau pour faciliter les développements.

Comme il n'y a pas une API OpenGL mais plusieurs, la seconde idée est que \\
  1 l'utilisateur requiert/choisie son API, \\
  2 on génère à la volée le wrapper, càd. un module contenant les constants et les fonctions de l'API souhaitée. \\

Les commandes sont encapsulées dans une classe dit ``Command Wrapper'' 

\section{Slide 17}

Dont je vais vous présenter un schéma de principe simplifié.

L'appelle d'une fonction en C, requiert trois étapes~: \\
 1 traduire les paramètres de Python vers C \\
 2 appeler la fonction \\
 3 traduire les valeurs retournées de C vers Python

Ces trois étapes ont chacune leur classe.

Lorsque on créé une fonction, le constructeur de la classe CommandWrapper, parcourt le prototype est
créé pour chacun une classe ParameterWrapper. En faîte il y a un ``if'', j'expliquerais pourquoi
plus tard.

L'appelle de la fonction OpenGL se fait via la méthode spéciale ``\_\_call\_\_'', qui est chargé de
mettre en oeuvre les trois étapes. Pour chaque paramètre passé par l'utilisateur, l'instance de
ParameterWrapper assure la conversion et retourne le cas échéant une instance de ToPythonConverter
pour effectuer la conversion inverse/retour. 

\section{Slide 18}

Ce mécanisme met en oeuvre une traduction automatique des paramètres que je vais vous présenter en
passant en revue les cinq types de paramètres.

Le principe de la traduction automatique consiste à exploiter les spécificités de l'API OpenGL, ce
qui est une approche particulière.

Pour les paramètres simples il y a juste à faire une conversion de Python vers C.

Pour les paramètres passé par référence, on souhaite cacher les détails du C et déplacer ces
paramètres dans la liste des valeurs retournés. Pour ce faire on créé en mémoire le paramètre et on
le retourne dans l'ordre à la suite des autres. (pointeur) On voit la traduction des prototypes ici.

Pour les pointeurs complexes, L'XML Registry ne donne pas de formule, on ne peut donc pas en l'état
faire grand chose. Hormis regarder dans l'ordre si le paramètre est de type chaîne de caractères et
sinon requérir un tableaux Numpy, pour lequel on vérifiera le type si il ne s'agit pas d'un pointeur
générique.

Il y a des solutions pour améliorer la situation~: \\
 1 on peut mettre en oeuvre un Pythonic Wrapper qui va encapsuler l'instance CommandWrapper \\
 2 compléter le fichier XML

\section{Slide 19}

Pour les tableaux en lecture seule, l'XML Registry est aussi incomplète car elle n'indique pas si la
taille est exprimé en nombre d'éléments ou en octet.

Logiquement on va supposer que les pointeurs génériques on une taille exprimée en octet et les
autres en nombre d'éléments.

Le ParameterWrapper gère trois cas de figure~: \\
  les chaîne de caractères, \\
  les tableaux Numpy, \\
  et les iterables qui sont plus confortable pour les petits tableaux (comme les uniforms).

Comme on connaît la taille des objets en Python, le wrapper va se charger de passer la bonne valeur
pour l'utilisateur.

\section{Slide 20}

À présent il nous reste plus cas regarder le cas des tableaux en écriture (et lecture).

Le wrapper gére quatre cas~: \\
 1 si le paramètre est de type générique, alors on requiert un tableaux Numpy, \\
 2 si le paramètre est une chaîne de caractères, alors on une requiert une taille et on créé une
 chaîne de caractères qui sera retourné par l'appelant. \\
 3 si l'utilisateur passe un tableaux Numpy, alors on se retrouve dans un cas similaire au premier. \\
 4 sinon on requiert une taille et on créé selon une valeur seuil soit une liste soit un tableaux
 Numpy qui sera retourné par l'appelant comme dans le second cas.  

\section{Slide 21}

Pour simplifier la vie du programmer, \\
cette interface dispose d'un utilitaire pour un interroger la GL API, ce qui utile pour valider les
concepts du wrapper.

La classe CommandWrapper dispose d'une fonction retournant le mode d'emploi de la fonction, lors d'une
session Ipython par exemple.

Elle dispose aussi d'un compteur d'appels pour tracer son application.

Au niveau de la gestion d'erreur qui n'est pas simple en C, le module dispose d'un context manager
pour lancer une exception si une erreur est survenu. On peut aussi vérifier cela lors d'un appel de
fonction en activant un kwargs.

\section{Slide 22}

Pour ce qui concerne l'API de haut niveau on souhaite essentiellement une surcouche pour faciliter
le développement d'application qui peut être différente selon le type d'application.

Ainsi qu'un moyen d'initialiser ou lire les uniforms dans les shader au moyen d'une interface
orienté objet. (varying binding)

\section{Slide 23}

Pour aller plus loin, \\
vous trouverez le code source et la documentation dans ce dépôt github.

Il y a encore du travail, \\
tel que peaufiner la GL API, \\
éventuellement retirer le wrapper ctypes au profit de cffi, \\
tester plus largement le wrapper et faire des benchmarks, \\
et enfin améliorer l'API de haut niveau.

Au vue de la taille de l'API OpenGL, 650 fonctions, il est évident qu'il est difficile de clamer
d'être conforme à 100 \% avec l'API OpenGL.

\section{Slide 24}

Je vais à présent vous faire une démonstration mettant en oeuvre cette interface.

Dans un premier temps je vais vous montrer un démonstrateur inclus dans le code source de l'interface.

Et ensuite un visualisateur de fichier DVI générer par LaTeX reposant sur le package Python PyDvi
que j'ai développé qui permet de lire ces fichiers. L'originalité de ce visualisateur est qu'il est
programmé en Python et met en oeuvre une accélération matériel par GPU.

\end{document}

%%% Local Variables: 
%%% mode: latex
%%% TeX-master: t
%%% End: 
