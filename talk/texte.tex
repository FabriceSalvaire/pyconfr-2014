\documentclass[12pt]{article}

\begin{document}

\section{Slide 1}

Bonjour, 

Je vais vous présenter une interface expérimentale pour lier OpenGL à Python,
que j'ai réaliser au cours de cette année.

\section{Slide 2}

Ma présentation va s'articuler en trois temps~:
  dans un premier temps je vais introduire l'API OpenGL et les Gpus,
  ensuite je vais vous parler de l'interface
  et finir par une démonstration.

\section{Slide 3}

Pour commencer nous allons regarder l'évolution des processeurs graphiques au cours du temps,

en 1970 les pionniers construisaient de grosses machines,
ensuite les stations de travail sont apparues,
puis les accélérateurs graphiques se sont démocratisés,
en 2000 Nvidia a commercialisé le premier GPU,
puis en 2006 sont apparus les processeurs génériques.

On observe une convergence, il n'y a plus de différence entre le monde professionnel et grand-publique,
la même technologie est utilisé de l'embarqué aux super-calculateurs.

\section{Slide 4}

Qu'est ce qui différencie un GPU d'un CPU ?

Ce n'est pas la même manière d'exploiter la surface de silicium comme vous le voir sur ces photos de
dies.

Il s'agit ...

\section{Slide 5}

À quoi servent les GPUs ?

On peut les utiliser pour la 3D, dans les moteurs de jeu, la CAO, la visualisation scientifique, et
plus récemment pour les interfaces graphiques et même la 2D.

\section{Slide 6}

L'API OpenGL est un standard ouvert gouverné par le groupe khronos qui permet de -piloter- un GPU.
et la seule API multi-platforme.
C'est l'API pour Linux et Android.
Il y a plusieurs API, orientée desktop, OpenGL ES pour l'embarqué et WebGL pour le web.

OpenGL n'est pas un monde idéal, mais il y a des initiatives pour une nouvelle génération.

\section{Slide 7}

OpenGL a été créé par Silicon Graphics il y a plus de 20 ans déjà. En 2008 son développement s'est
accéléré avec l'avènement du pipeline programmable livré avec la V3. Mais il y a des concurrents.

Comme souvent on observe un déphasage due à l'inertie au changement. Il y a un déphasage entre
OpenGL, le matériel, le parc machine, et le programmeur.

Lorsqu'on développe une application avec OpenGL, une question fondamental est de savoir qu'elle API choisir.

\section{Slide 8}

L'API OpenGL est découpé en extensions où chacune regroupe un ensemble de nouvelle fonctions avant
leurs intégrations dans l'API core.

Mesa est à l'origine une implémentation logicielle d'OpenGL, aujourd'hui c'est un composant
essentiel pour OpenGL sous Linux pour les drivers open source.

Le graphique représente le statu(s) actuel de Mesa pour le GPU d'Intel.
On voit que l'on est à deux doigts d'OpenGL V4.

Aujourd'hui seul le driver propriétaire de Nvidia implémente la quasi totalité de l'API OpenGL.

\section{Slide 9}

Aujourd'hui OpenGL se résume à un langage de calcul générique GLSL qui est mis en oeuvre dans des programmes
appelés shader et des fonctions spécifiques au graphisme.

Les paramètres d'un shader sont des constantes appelés uniform qui peuvent être des floatants, des
matrices, ou des indexes représentant une texture. Les données variables d'un shader sont des
vecteurs de une à quatre dimensions.

Ces vecteurs sont issues d'un flux de donnés qui représente des vertex de primitive graphiques tel
que des lignes et des triangles qui sont interpolés en pixels.  

\section{Slide 10}
\section{Slide 11}
\section{Slide 12}
\section{Slide 13}
\section{Slide 14}
\section{Slide 15}
\section{Slide 16}
\section{Slide 17}
\section{Slide 18}
\section{Slide 19}
\section{Slide 20}
\section{Slide 21}
\section{Slide 22}
\section{Slide 23}
\section{Slide 24}
\section{Slide 25}

\end{document}

%%% Local Variables: 
%%% mode: latex
%%% TeX-master: t
%%% End: 
