%%%%%%%%%%%%%%%%%%%%%%%%%%%%%%%%%%%%%%%%%%%%%%%%%%%%%%%%%%%%%%%%%%%%%%%%%%%%%%%%%%%%%%%%%%%%%%%%%%%%%%%%%%%%%%%%%%%%%%%%

% keep tikz
% tikzpicture ???

% Redefinition, symbol included in link:
\let\orighref\href
%\renewcommand{\href}[2]{\orighref{#1}{#2\,\resizebox{!}{1.mm}{\faExternalLink}}} % lualatex
% \pgfdeclareimage[height=1mm]{externalLink}{images/external-link-small.png}
\renewcommand{\href}[2]{%
\orighref{#1}{#2\,%
  \begin{tikzpicture}[scale=.1, mystyle/.style={line width=.25pt, line cap=round, rounded corners=.5pt}]
    \draw[mystyle] (1,.4) -- (1,0) -- (0,0) -- (0,1) -- (.7,1);
    \fill (.8,1.2) -- ++(.5,0) -- ++(0,-.5);
    \draw[mystyle] (.6,.4) -- (1.1,1.);
  \end{tikzpicture}%
%\pgfuseimage{externalLink}
}}

%%%%%%%%%%%%%%%%%%%%%%%%%%%%%%%%%%%%%%%%%%%%%%%%%%%%%%%%%%%%%%%%%%%%%%%%%%%%%%%%%%%%%%%%%%%%%%%%%%%%%%%%%%%%%%%%%%%%%%%%

\newcommand{\ptr}{\textasteriskcentered}
\newcommand{\code}[1]{\texttt{#1}}

%%%%%%%%%%%%%%%%%%%%%%%%%%%%%%%%%%%%%%%%%%%%%%%%%%%%%%%%%%%%%%%%%%%%%%%%%%%%%%%%%%%%%%%%%%%%%%%%%%%%%%%%%%%%%%%%%%%%%%%%

\newcommand{\colorR}[1]{{\color{red!80!black}#1}}
\newcommand{\colorG}[1]{{\color{green!80!black}#1}}
\newcommand{\colorB}[1]{{\color{blue!80!black}#1}}
\newcommand{\bgR}{red!20!white}

\newcommand{\boxR}[1]{\colorbox{red!20!white}{#1}}
\newcommand{\boxG}[1]{\colorbox{green!20!white}{#1}}
\newcommand{\boxB}[1]{\colorbox{blue!20!white}{#1}}

%%%%%%%%%%%%%%%%%%%%%%%%%%%%%%%%%%%%%%%%%%%%%%%%%%%%%%%%%%%%%%%%%%%%%%%%%%%%%%%%%%%%%%%%%%%%%%%%%%%%%%%%%%%%%%%%%%%%%%%%

\pgfdeclareimage[height=2cm]{FunnyPython}{images/funny-python.png}
\pgfdeclareimage[height=10mm]{FunnyPythonSmall}{images/funny-python.png}
\pgfdeclareimage[height=1cm]{OpenGLLogo}{images/khronos-logos/OpenGL/OpenGL_500.jpg}

%%%%%%%%%%%%%%%%%%%%%%%%%%%%%%%%%%%%%%%%%%%%%%%%%%%%%%%%%%%%%%%%%%%%%%%%%%%%%%%%%%%%%%%%%%%%%%%%%%%%%%%%%%%%%%%%%%%%%%%%

% raise !
\newcommand{\attention}{%
  \begin{tikzpicture}
    \node [fill=yellow!80, isosceles triangle, shape border rotate=90, inner sep=.1pt, rounded corners=1.5pt] {\textbf{!}};
  \end{tikzpicture}}

% check beamer code
\newcommand{\numberItem}[1]{
  \begin{tikzpicture}
    \node[circle, fill=blue!70!black, draw=none, inner sep = .5pt] at (0,0)
    {\textcolor{white}{\scriptsize #1}};
  \end{tikzpicture}}

%%%%%%%%%%%%%%%%%%%%%%%%%%%%%%%%%%%%%%%%%%%%%%%%%%%%%%%%%%%%%%%%%%%%%%%%%%%%%%%%%%%%%%%%%%%%%%%%%%%%%%%%%%%%%%%%%%%%%%%%
%%%
%%% Local Variables: 
%%% mode: latex
%%% TeX-master: "master"
%%% End: 
%%%
%%%%%%%%%%%%%%%%%%%%%%%%%%%%%%%%%%%%%%%%%%%%%%%%%%%%%%%%%%%%%%%%%%%%%%%%%%%%%%%%%%%%%%%%%%%%%%%%%%%%%%%%%%%%%%%%%%%%%%%%
